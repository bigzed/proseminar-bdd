\section{Einleitung}
  BDD ist eine Weiterentwicklung von Test Driven Development und Verkn"upfung 
  mit anderen bestehenden agilen Software Processen wie Domain Driven Design 
  und Object Oriented Analysis.\\
  Programmierer die das erste mal mit TDD in Ber"uhrung kommen brauchen eine 
  lange aufw"arm Phase bis sie effektiv mit diesem Konzept arbeiten k"onnen. 
  Denn bei der Verfolgung dieses Konzeptes stellen sich als erstes zwar simpel 
  erscheinende doch sehr wichtige Fragen.\\
  \begin{itemize}
    \item Was soll getestet werden?
    \item Wo f"angt man mit dem Testen an?
    \item Wieviel sollte ein Test "uberhaupt testen?
    \item Wie sollten Tests benannt sein?
    \item Wie sollte verstanden werden warum ein Test fehlschl"agt?
  \end{itemize}
  Diese Fragen f"uhren zu einer starken Diversit"at zwischen den Umsetzungen
  der Idee von TDD zwischen vielen Programmierer\_innen und erschwert es das Konzept
  schnell und klar an andere Entwickler\_innen weiter zu vermitteln.\\
  Es braucht also eine klar Struktur von 'Best Practises' f"ur TDD um den 
  Umgang mit diesem Prozess zu vereinheitlichen und Anf"angern die M"oglichkeit
  zu geben auch ohne Vorkenntnisse schnell in bestehende Projekt mit diesem 
  Prozess einzusteigen.\\
  Genau diese Probleme, die von Programmierer\_innen die Erfahrung im Umgang mit TDD 
  haben meist umgangen werden indem sie "ahnliche konzeptionelle Methoden wie 
  BDD verfolgen, wurden von Dan North\cite{North:2006} in seinem 
  Artikel erstmals zusammengefasst.\\
  Genau auf diese Konzepte werde ich n"aher eingehen.