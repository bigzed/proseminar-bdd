\section{Zusammenfassung}
  Zwar scheint es so das BDD nur eine Sammlung von Tools und 'Best Practises' 
  sind die speziell an 'Test-First' Enwicklungszyklen angepasst sind, allerdings
  macht genau dies den Vorteil aus. Wie in der Einleitung schon erw"ahnt ist
  genau dies der Mangel der bei TDD besteht.\\
  Einen wissenschaftlichen Vergleich auf Basis von empirischen Studien existiert
  zwar noch nicht zwischen BDD und TDD, allerdings k"onnen wir aus den 
  existierenden Studien zu TDD einige Schl"usse ziehen.\\
  Studien wie 'Implications of Test-Driven-Development' \cite{Kaufmann:Janzen:2003}
  zeigen das allein TDD schon Vorteil in der Entwicklung bringt, es erh"oht die 
  Produktivit"at der Entwickler\_innen und verbessert die Codequalit"at. 
  Auch Studien von IBM und Microsoft \cite{IBM:Microsoft:2008} zeigen das allein
  durch TDD die Fehlerquote im Quellcode um 40\% oder mehr sinkt und der 
  Zeitaufwand zur Erstellung des Quellcodes nur um 15\% - 35\% steigt.
  Dabei ist au"serdem noch in Betracht zuziehen, dass durch die bessere Wartbarkeit
  des Quellcodes dieser mehr Aufwand sich noch akkumulieren wird.\\
  Ein weiterer Punkt den Studien zu Test-Driven-Development hervorgebracht 
  haben, ist das die entstehende Software besser zu Erweitern ist und ein 
  besseres Design als Test-Last Software haben.\cite{Janzen:2006}\\
  Doch in allen Studien wird auch festgestellt das Entwickler\_innen anfangs
  Schwierigkeiten haben diese neue Art des Entwickelns schnell auf zu greifen,
  da das Entwickeln von Tests f"ur Quellcode der noch nicht existiert eine
  Gedankenh"urde darstellt. Nun k"onnen wir allerdings annehmen das durch
  die klaren Strukturen von BDD diese H"urde minimiert wird. Somit ist es zu 
  bevorzugen den Entwicklungsablauf besonders in Agilen Prozessen auf BDD
  umzustellen.\\